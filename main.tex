\documentclass{article}
\usepackage[utf8]{inputenc}

\title{Dokumentacja}
\author{Kamil Zachara}

\usepackage{natbib}
\usepackage{graphicx}
\usepackage[margin=0.5in]{geometry}

\usepackage{listings}
\usepackage{color}

\definecolor{dkgreen}{rgb}{0,0.6,0}
\definecolor{gray}{rgb}{0.5,0.5,0.5}
\definecolor{mauve}{rgb}{0.58,0,0.82}

\lstset{frame=tb,
  language=Java,
  aboveskip=3mm,
  belowskip=3mm,
  showstringspaces=false,
  columns=flexible,
  basicstyle={\small\ttfamily},
  numbers=none,
  numberstyle=\tiny\color{gray},
  keywordstyle=\color{blue},
  commentstyle=\color{dkgreen},
  stringstyle=\color{mauve},
  breaklines=true,
  breakatwhitespace=true,
  tabsize=3
}

\begin{document}

\maketitle

\section{Opis projektu}
Projekt ten jest implementacją gry "Kółko krzyżyk" w oparciu o architekturę klient-serwer.

\begin{figure}[h!]
\centering
\includegraphics[scale=0.8]{menu.png}
\caption{Menu programu}
\end{figure}

\section{Instrukcja}
Menu Programu pozwala na wybór figury którą będziemy umieszczać na planszy (w przeciwnieństwie do klasycznej wersji gry, wybór X nie oznacza że zaczynamy jako gracz 2)
następną z opcji jest określenie kolejności gry(najprościej mówiąc, określamy kto zaczyna pierwszy, gracz lub komputer)
\newpage
\section{Gra z komputerem}
Algorytm pracy wbudowanego w grę AI wybiera losowe wolne pole, w którym stawia przypisany mu symbol.

\subsection{Kod Ai}
\begin{lstlisting}
    public Board aiMoveFunction(AiMoveDto move) {
        Random rand = new Random();
        for(int i=0;i<board.gameBoard.length;i++){
            for(int j=0;j<board.gameBoard[i].length;j++){
                if(board.gameBoard[i][j]==null){
                    while(true){
                        int x = rand.nextInt(3);
                        int y = rand.nextInt(3);
                        Character tmp = board.gameBoard[x][y];
                        System.out.println(x + " " + y + " " + tmp);
                        if (tmp == null){
                            System.out.println("#END");
                            board.gameBoard[x][y] = move.getMark();
                            break;
                        }
                    }
                    return board;
                }
            }
        }
        return null;
    }
\end{lstlisting}


\section{Usługi serwera}

\subsection{result()}
URL Usługi: 127.0.0.1:8080/tictactoe/result

Typ żadania: GET

Parametry wejściowe(JSON): Brak

Parametry Wyjścowe(JSON):

\begin{lstlisting}
["x", null, "o",
"x", "x", "o",
"o", "o", null]
\end{lstlisting}

Kod odpowiedzi serwera: 200



\subsection{restartGameFunction()}

URL Usługi: 127.0.0.1:8080/tictactoe/reset-game 

Typ żadania: POST 

Parametry wejściowe(JSON): Brak 

Parametry Wyjścowe(JSON): Brak 

Kod odpowiedzi serwera: 200 

\newpage



\subsection{UserMoveDto()}
URL Usługi: 127.0.0.1:8080/tictactoe/set-field-by-user

Typ żadania: POST

Parametry wejściowe(JSON): 

\begin{lstlisting}
{
    "fieldX": "0",
    "fieldY": "0",
    "mark": "x"
}
\end{lstlisting}

Parametry Wyjścowe(JSON): Brak 

Kod odpowiedzi serwera: 200

\subsection{AiMoveDto()}
URL Usługi: 127.0.0.1:8080/tictactoe/set-field-by-ai

Typ żadania: POST

Parametry wejściowe(JSON): 

\begin{lstlisting}
{
    "mark": "o"
}
\end{lstlisting}

Parametry Wyjścowe(JSON): Brak 

Kod odpowiedzi serwera:

- 200 gdy zostało wypełnione pole

- 400 gdy tablica jest wypełniona w całości



\section{Narzędzia wykorzystane w projekcie}
\subsection{InteliJ IDEA}
\begin{figure}[h!]
\centering
\includegraphics[scale=0.5]{intelij.png}
\end{figure}
Do stworzenie projektu skorzystaliśmy ze środowiska programistycznego od JetBrains.
\subsection{Postman}
\begin{figure}[h!]
\centering
\includegraphics[scale=0.5]{postman.png}
\end{figure}
Do testowania API użyty został program Postman.
Pozwala na wykonywanie zapytań do aplikacji, następnie zwracając ich wyniki, dzięki czemu .jest niezwykle użyteczna podczas testowania funkcjinalności aplikacji.
\newpage
\subsection{Maven}
\begin{figure}[h!]
\centering
\includegraphics[scale=0.5]{maven.png}
\end{figure}
Narzędzie automatyzujące budowę oprogramowania na platformę Java.
Automatyzuje proces dodawania wtyczek do programu, pobierając je przy pierwszym ich użyciu

\subsection{Bitbucket}
\begin{figure}[h!]
\centering
\includegraphics[scale=0.5]{bitbucket.png}
\end{figure}
Usługa hostingowa, używana do tworzenia kodu źródłowego.
Pozwala na bezpieczne "przechowywanie kodu" w chmurze, oraz podział aplikacji na poszczególne "branche" co pozwala na rozwijanie aplikacji w bezpieczniejszy sposób gdyż zmiany jednej rzeczy nie wpływają na główną aplikację, do momentu dołączenia zmian do brancha głównego

\subsection{Lombok}
Biblioteka, automatyzująca tworzenie setterów i getterów







\section{Podział prac}
\begin{center}
Server - Patryk Starzycki

Client - Maciej Szostak

Dokumentacja - Kamil Zachara
\end{center}

\end{document}


